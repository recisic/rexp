\documentclass{article}
\usepackage[left=30mm,right=30mm,top=30mm,bottom=30mm]{geometry}
\usepackage{kotex}
\usepackage{lipsum}
\usepackage{setspace}
\setlength{\columnsep}{20pt}
\usepackage{multicol}
\begin{document}
	\setstretch{1.25}
\begin{center}
	\Huge \textsf{Homework 1 Report}\\
	\Large \textsf{4190.306~ Automata Theory}\\ \vspace{0.15cm}
	\large \textsf{자연과학대학 화학부 $\cdot$ 2017--19871 남준오}\\
	\normalsize
	\vspace{0.5cm}
\end{center}

\begin{multicols}{2}

\section{헤더 파일}
\begin{itemize}
	\item \texttt{iostream}\\
	\item \texttt{fstream}\\파일 입출력을 위해 사용
	\item \texttt{string}\\정규 표현식이나 postfix 표현 등의 문자열을 다루기 위해 사용
	\item \texttt{stack}\\Postfix 표현을 만들 때 쓰인 operator stack이나 NFA를 만들 때 쓰인 intermediate NFA stack을 구현할 때 사용
	\item \texttt{set}\\$\Delta(q, w)$의 결과가 집합으로 주어지므로 해당 집합을 구현하기 위해 사용
\end{itemize}

\section{동작 원리}
\lipsum[1]

\section{예제 및 출력 결과}
\lipsum[1]

\end{multicols}

\end{document}​​​​​​​​​​​​​​​